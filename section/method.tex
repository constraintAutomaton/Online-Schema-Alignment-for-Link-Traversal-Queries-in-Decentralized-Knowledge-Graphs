\section{Online Schema Aligmnent}

\subsection{Schema Alignment}
Schema alignment requires the definition of a set of rules.
In this work, we draw inspiration from the \gls{sssom} specification~\cite{sssom_website, matentzoglu2022simple} to structure and formalize these rules.
An example of the resulting serialization is presented in Figure~\ref{fig:ruleSet}.

For the purpose of this work, we developed a simplified version of \gls{sssom} and introduced the concept of a \emph{subweb} (see Section~\ref{sec:dkg}) to specify the domain of applicability of the schema alignment rules.
In other words, these rules apply only within the designated subweb.
We opted for this design for several practical reasons.

Our system operates under the open-world assumption, in which multiple new \glspl{kg} and schema alignment rules may be discovered during processing.
To ensure robustness, the system must prevent infinite recursive rule applications.\footnote{With our schema alignment entailment, contradictions are not possible.}
When such situations occur, it should ideally resolve them without aborting the entire computation.
Since rules are applied within specific domainsthe likelihood of infinite recursion is significantly reduced.
In cases where infinite recursion is detected or of overlap of subwebs, a simple resolution strategy is employed: the newly generated schema alignment rule set is rejected.

Another important consideration is the potential misuse, overspecification, and underspecification of ontology terms~\cite{abedjan2012reconciling, lorey2011rdf}.
In our use case, the goal is to enable querying data across the web under the assumption that data providers best understand their own data models and the intended entailments.
However, we also allow query issuers to define additional rules that may interact with those provided by data publishers.
This is justified by the fact that query issuers can more easily track and manage the consequences of their own entailment definitions, whereas it would be unrealistic to expect data providers to anticipate the broader implications of their rules beyond the scope of their own datasets or use cases.

\Gls{sssom} defines multiple predicates for aligning ontology terms with varying levels of precision.
For instance, \texttt{owl:sameAs} indicates that two instances are identical, whereas \texttt{skos:broadMatch} denotes that two instances are broadly related.\footnote{Prefix definitions are omitted for brevity.}
In addition, \gls{sssom} specifies a set of rules to support transitivity, chaining, inversion, and generalization of alignments~\cite{sssom_website}.~\footnote{\url{https://mapping-commons.github.io/sssom/chaining-rules/}}
These rules operate on the predicates used for defining alignments; for example, transitivity cannot be applied to \texttt{skos:relatedMatch}, nor can chaining be applied to rules involving \texttt{owl:equivalentClass} and \texttt{rdfs:subClassOf}.
This mechanism enables reasoning over the level of precision of mappings when resolving schema alignments.

\begin{figure}
    \centering
    \lstinputlisting[basicstyle=\footnotesize\ttfamily]{code/rules.ttl}
    \caption{
       The example of a serialization of the mapping rules in a subweb.
       Those rules map between two versions of a vocabulary.
       }
    \label{fig:ruleSet}
\end{figure}

\subsection{Discovery and Processing of Alignment Rules}
We define $AR_t$ as the set of mapping between subwebs $S$ and schema algiment rules $R$ at the time $t$.
\begin{equation} \label{eq:schema_alignment_scoped}
    AR = \left\{ S_i \mapsto R_i| 1 \ge i \le n \right\} \cup \left\{ S_j \mapsto R_j| 1 \ge j \le n_t \right\}
\end{equation}
Where $n$ is the number of query issuer defined alignment rules and $n_t$ is the number of discovered alignment rules at the time $t$.

\subsubsection{Software Architecture}
test
