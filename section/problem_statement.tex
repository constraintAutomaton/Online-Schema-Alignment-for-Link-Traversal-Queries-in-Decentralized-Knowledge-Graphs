\section{Problem Statement}
The problem motivating this demo paper is that, in many use cases, different data providers express semantically equivalent properties in heterogeneous ways, which causes difficulties when performing link traversal queries.
Some of these inconsistencies may appear trivial, for instance, the use of both \gls{https} and \gls{http} vocabularies, yet they still require query adaptations, since the query issuer often intends to treat both vocabularies as equivalent.
In such simple cases, one might address the issue by adding \texttt{UNION} clauses to the query.

However, this approach has several limitations.
First, it assumes that, within a large and decentralized network, the query issuer is aware of all relevant alignment rules.~\footnote{There are also considerations regarding query size and interpretability.}
Second, it presumes that these alignment rules are globally applicable across every encountered subweb, which is unrealistic given the inherent complexity of ontology design~\cite{abedjan2012reconciling, lorey2011rdf}.
A further challenge arises when rule chaining is required: this either necessitates materializing intermediate \glspl{kg} through multiple \texttt{CONSTRUCT} queries or performing rule backchaining to materialize a final query~\cite{maarten2023pod}.
Neither strategy is compatible with \gls{ltqp}.
Executing multiple queries disrupts traversal policies, as the query engine would need to maintain awareness of resources already traversed by other queries.
Conversely, materializing only a final query through backchaining fails to account for intermediate queries that may be essential for discovering and traversing the required subwebs.

More complex cases involve deeper semantic mismatches.
For example, different providers may conceptualize a response to a post as either a "reply" or a "comment".
Similarly, some may represent attributes such as gender or city names as literals, while others may use \glspl{iri}.~\footnote{See Section~\ref{sec:kg}}
