\section{Introduction}

The integration of multiple knowledge bases poses substantial challenges across a wide range of applications.
Such integration is often driven by diverse motivations, including the need to collaborate with heterogeneous stakeholders who maintain independent data sources,
the pursuit of decentralized system architectures, and the ambition to enable cross-domain applications within specialized domains.

The integration of multiple knowledge bases presents several challenges.
A first challenge concerns scalability: due to the size of the data or privacy constraints, it is often impractical to materialize all knowledge bases within a single database system.
In such cases, paradigms such as federated querying~\cite{buil2013federating} or link traversal-based querying~\cite{Hartig2012} can effectively address these limitations.
Another major challenge arises from heterogeneity in knowledge representation or vocabulary, particularly in the context of \gls{kg}, which is the focus of this paper.
In many use cases, query issuers expect to formulate their queries under the assumption of a single, consistent knowledge representation.

This demo paper focuses on online schema alignment during \gls{ltqp}~\cite{Hartig2012}, enabling the execution of SPARQL queries~\cite{sparql_specification} over distributed networks.
We propose a mechanism to dynamically retrieve remote schema alignment rules in order to align the target datasets with the user-issued query.
In addition, we introduce a complementary mechanism that allows users to supply their own schema alignment rules alongside a query.
These user-provided rules can either extend those discovered in the network or align queries that use alternative vocabularies with the target network.
The remainder of this paper is organized as follows: we first present the problem statement, followed by preliminaries and the proposed approach.
We then describe the demonstrator, discuss related work, and conclude the paper.
