\section{Preliminary}

\subsection{Reachability Criteria~\cite{traveling_map_ltqp}}

LTQP defines completeness on the traversal of links instead of the query results~\cite{Hartig2012}.
To formalize the completeness of queries, \emph{Reachability criteria}~\cite{Hartig2012} have been formalized.
Reachability criteria are boolean functions ($c_i$) restricting the dereferencing of links from the internal data source of the query engine.
They take as parameters an RDF triple $t$ from an internal triple store, a dereferenceable \gls{iri} $iri$ from $t$, and a union of conjunctive queries $Q$.
If $c_i$ returns $true$, the query engine must dereference $iri$.
More formally
\begin{equation}\label{eq:reachabilityCriteria}
c_i(t, iri, Q) \rightarrow \left\{ \mathrm{true}, \mathrm{false} \right\}
\end{equation}

\subsection{Decentralized Knowledge Graphs and Subweb~\cite{traveling_map_ltqp}}\label{sec:dkg}

We define a DKG as a KG $G$ materialized in a network of resources $R$.
A resource $r_i \in R$ is mapped to a KG $g_i \subseteq G$, which is a set of triples~\cite{w3ConceptsAbstract}.
We denote this mapping $r_i \mapsto_{\mathcal{G}} g_i$.
A resource is mapped and exposed by an \gls{iri} $iri_i \in \mathcal{I}$ denoted by $iri_i \mapsto_{\mathcal{R}} r_i$ where $\mathcal{I}$ is the set of all \glspl{iri}.
The network forms a graph where the resources $r_i$ are the nodes and the $iri_j$ are directed edges starting from $r_i$ to $r_j$.
The $iri_j$ are RDF terms in the triples in $g_i$.
$G$ is formed by the union of all the $g_i$ mapped to a resource in the network.
A subweb is a (sub)DKG defined by the KG derived from a set of \glspl{iri} controlled by a data provider.

\subsection{Link Traversal Query Processing}
