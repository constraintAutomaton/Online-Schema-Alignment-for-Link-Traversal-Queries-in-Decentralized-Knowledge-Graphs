\section{Preliminary}
This section present some preliminary notions to understand our approach.

\subsection{Knowledge Graph and \gls{rdf}}

A \gls{kg} is a representation of knowledge in the form of a graph expressed as a set of triples~\cite{w3ConceptsAbstract}.
\Gls{rdf} triples $t = (s, p, o)$ are tuples composed of three terms~\footnote{Definition inspired by ~\citetitle{traveling_map_ltqp}~\cite{traveling_map_ltqp}}:
a subject $s \in \mathcal{I} \cup \mathcal{B}$,
a predicate $p \in \mathcal{I}$,
and an object $o \in \mathcal{I} \cup \mathcal{B} \cup \mathcal{L}$.
Here, $\mathcal{I}$, $\mathcal{B}$, and $\mathcal{L}$ denote, respectively, the sets of all possible \glspl{iri}, blank nodes, and literals.

\subsection{Decentralized Knowledge Graphs and Subweb~\cite{traveling_map_ltqp}}\label{sec:dkg}

We define a DKG as a KG $G$ materialized in a network of resources $R$.
A resource $r_i \in R$ is mapped to a KG $g_i \subseteq G$, which is a set of triples~\cite{w3ConceptsAbstract}.
We denote this mapping $r_i \mapsto_{\mathcal{G}} g_i$.
A resource is mapped and exposed by an \gls{iri} $iri_i \in \mathcal{I}$ denoted by $iri_i \mapsto_{\mathcal{R}} r_i$ where $\mathcal{I}$ is the set of all \glspl{iri}.
The network forms a graph where the resources $r_i$ are the nodes and the $iri_j$ are directed edges starting from $r_i$ to $r_j$.
The $iri_j$ are \gls{rdf} terms in the triples in $g_i$.
$G$ is formed by the union of all the $g_i$ mapped to a resource in the network.
A subweb is a (sub)DKG defined by the KG derived from a set of \glspl{iri} controlled by a data provider.

\subsection{Link Traversal Query Processing}

\Gls{ltqp} is a query paradigm that consist of exploiting the fact that \glspl{iri} in triples can be dereferenced to find \gls{kg} that should provide more useful information about a term~\cite{linked_data}.
The processing of those queries consist of starting with an initial empty knowledge base (a \gls{kg} inside of a database), a set of initial links and a set of \textit{reachability criteria}~\cite{Hartig2012}.
The query processing start by dereferencing~\footnote{By dereferencing we mean performing an \gls{http} GET on an \gls{iri}} the initial links and inserting the \gls{kg} from those resources inside the internal knowledge base.
The next step consist of traversing the network using the triple inside of the knowledge base using the reachability criterion.
Reachability criterion are boolean function that returns $true$ iff an \gls{iri} from a triple respect a condition defined by the engine.
Example of such criterion are following the object term of triples having a predicate term with a specific \gls{iri} or the criterion \texttt{Cmatch}~\cite{hartig2016walking} which consist of dereferencing
\gls{iri} related to the variable of the query.
For each triple inside of the knowledge base each reachability criterion determine which new \glspl{iri} to follow.
In concurence the query is process in the internal knowledge base and return the results in a streaming way, such that the traveral does not have to terminate before the user is receiving results.
