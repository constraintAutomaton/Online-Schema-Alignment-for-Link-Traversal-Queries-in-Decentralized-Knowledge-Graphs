\section{Demonstration}

In this section, we present a demonstration of our system.

We implemented an online schema alignment system using the link traversal version of the Comunica~\cite{taelman2018comunica} query engine.
Both our demonstration~\footnote{\url{https://github.com/onlineSchemaAlignmentLTQP/demo}} and implementation~\footnote{\url{https://www.npmjs.com/package/query-sparql-link-traversal-solid-schema-alignment}} are open source and publicly available.
Users can download the implementation to integrate it as a library or use its command-line interface to execute queries.

Our demonstration scenario is based on a social media network built using the SolidBench benchmark~\cite{Taelman2023}.
This social media network is decentralized, meaning that users store their data in personal data vaults called \emph{pods}, thus there are no centralized endpoint to query the user information.
For the purpose of this demonstration, we modified the user data such that different users express similar information using different vocabularies.
Additionally, each user exposes a file that describes their schema alignment rules, following the format illustrated in Figure~\ref{fig:ruleSet}.

The modifications include:
\begin{itemize}
    \item Using the original vocabulary.
    \item Using an alternative version of the vocabulary.
    \item Using the HTTPS variant of the vocabulary.
    \item Using an \gls{iri} derived from literal values, e.g., ``male'', ``female'', ``Firefox'', ``Chrome''.
    \item Describing the alignment between certain \glspl{iri} from the original vocabulary and the corresponding terms in Wikidata and DBpedia.
\end{itemize}

The demonstrator is a web-based application that allows users to execute either the proposed queries or arbitrary custom queries.
Some pop-ups are displayed to provide context for the demonstration and guide the user.
They can be turned off with a button and are shown automatically only on the first visit.
The demonstrator is hosted online~\footnote{TBD} and can also be deployed using a Docker Compose~\footnote{https://docs.docker.com/compose/} setup, which automatically generates the dataset, instantiates the servers, and launches the web application.
Queries can be run over different network configurations: the base network or variants that use alternative vocabularies.
The proposed queries are typical social media-related queries, such as retrieving information about a user, identifying the forums where a user has posted, and finding posts liked by specific users.
For each execution, the system provides detailed feedback, including the query results, execution time, and the alignments established during query processing, along with their associated subwebs.
Users can also define custom schema alignment rules and specify the subwebs to which these rules apply.
Additionally, the demonstrator includes a set of pre-defined queries using alternative vocabularies, together with the corresponding alignment rules that map them to the base SolidBench network.
Furthermore, users can explore and compare both network configurations through a web interface that presents them in a file-based navigation view.
We also provide a visual representation of the base network's data model, along with the set of alignment rules applied to the network variants that use different vocabularies.
